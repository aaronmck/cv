%%% A template to produce a nice-looking Curriculum Vitae.
%%% Original by Kieran Healy <kjhealy@gmail.com>, tweaked by Heather Miller <heather.c.miller@gmail.com>
%%% Most recent version is at http://github.com/heathermiller/cv
%%%
%%% ------------------------------------------------------------------------
%%% Requirements (should be included in a modern tex distribution):
%%% ------------------------------------------------------------------------
%%% xelatex
%%% fontspec.sty
%%% hyperrref.sty
%%% xunicode.sty
%%% color.sty
%%% url.sty
%%% fancyhdr.sty
%%%
%%% ------------------------------------------------------------------------
%%% Optional
%%% ------------------------------------------------------------------------
%%% git
%%% vc.sty
%%% revnum.sty
%%% Fonts
%%%
%%% ------------------------------------------------------------------------
%%% Note
%%%------------------------------------------------------------------------
%%% Because this is a hand-tweaked file, be on the look out for \medksip,
%%% \bigskip and \newpage commands here and there, which are used to balance
%%% the layout or avoid widows & orphans, etc. You should of course add or
%%% remove these as needed.
%%%------------------------------------------------------------------------

\documentclass[9pt]{article}

%%%------------------------------------------------------------------------
%%% Metadata
%%%------------------------------------------------------------------------

%% Change as needed. Or just add me as a coauthor. Only some of these are
%% used below in the hyperref declaration and address banner section.
\def\myauthor{Heather Miller}
\def\mytitle{Vita}
\def\mycopyright{\myauthor}
\def\mykeywords{}
\def\mybibliostyle{plain}
\def\mybibliocommand{}
\def\mysubtitle{}
\def\myaffiliation{EPFL}
\def\myaddress{Faculty of Computer, Communication, and Information Sciences}
\def\myemail{heather.miller@epfl.ch}
\def\myweb{http://heather.miller.am}
\def\myfax{+41 21 693 66 60}
\def\myphone{+41 78 625 20 23}
\def\myversion{}
\def\myrevision{}


\def\myaffiliation{EPFL}
\def\myauthor{Heather Miller}
\date{} % not used (revision control instead)
\def\mykeywords{Heather, Miller, Heather Miller, Vita, CV, Resume, Scala, Programming Languages}

%%%------------------------------------------------------------------------
%%% Git version tracking
%%%------------------------------------------------------------------------

%% If you don't use git or the vc package (from CTAN), comment this out.
%% If you comment it out, be sure to remove the \rfoot comment below, too.
\immediate\write18{sh ./vc}
%%% This file has been generated by the vc bundle for TeX.
%%% Do not edit this file!
%%%
%%% Define Git specific macros.
\gdef\GITHash{e0824d783a88eb464395fd836fc8cb9d666bcae5}%
\gdef\GITAbrHash{e0824d7}%
\gdef\GITParentHashes{9523c1e0762ed00ce249602437fcb0106493473d}%
\gdef\GITAbrParentHashes{9523c1e}%
\gdef\GITAuthorName{Heather Miller}%
\gdef\GITAuthorEmail{heather.miller@epfl.ch}%
\gdef\GITAuthorDate{2015-01-27 14:21:31 +0100}%
\gdef\GITCommitterName{Heather Miller}%
\gdef\GITCommitterEmail{heather.miller@epfl.ch}%
\gdef\GITCommitterDate{2015-01-27 14:21:31 +0100}%
%%% Define generic version control macros.
\gdef\VCRevision{\GITAbrHash}%
\gdef\VCAuthor{\GITAuthorName}%
\gdef\VCDateRAW{2015-01-27}%
\gdef\VCDateISO{2015-01-27}%
\gdef\VCDateTEX{2015/01/27}%
\gdef\VCTime{14:21:31 +0100}%
\gdef\VCModifiedText{\textcolor{red}{with local modifications!}}%
%%% Assume clean working copy.
\gdef\VCModified{0}%
\gdef\VCRevisionMod{\VCRevision}%


%%%------------------------------------------------------------------------
%%% Required style files
%%%------------------------------------------------------------------------

\usepackage{url,fancyhdr}
\usepackage[ampersand]{easylist}
%%\usepackage{revnum} % for reverse-numbered publications (revnumerate environment) if needed.

%% needed for xelatex to work
\usepackage{fontspec}
\usepackage{xunicode}

%% color for the links
% \usepackage[usenames,dvipsnames]{color}
\usepackage[usenames,dvipsnames]{xcolor}
\definecolor{DarkBlue}{HTML}{265B8C}

%% hyperlinks
\usepackage[xetex,
	colorlinks=true,
	urlcolor=DarkBlue,
	plainpages=false,
  	pdfpagelabels,
  	bookmarksnumbered,
  	pdftitle={\mytitle},
  	pagebackref,
  	pdfauthor={\myauthor},
  	pdfkeywords={\mykeywords}
  	]{hyperref}

\usepackage{marvosym}

% \usepackage{showframe}
\usepackage[width=4.825in,top=1.7in]{geometry}

%%%------------------------------------------------------------------------
%%% Document
%%%------------------------------------------------------------------------
\begin{document}

%% Choose fonts for use with xelatex
%% Minion and Myriad are widely available, from Adobe.
%% Pragmata is available to buy at http://www.fsd.it/fonts/pragma.htm
%% and is worth every penny. Any good monospace font will work fine, though.
%% Consolas or inconsolata are good alternatives.
\setromanfont[Mapping={tex-text},Numbers={OldStyle},Ligatures={Common}]{Minion Pro}
\setsansfont[Mapping=tex-text,Colour=AA0000]{Myriad Pro}
\setmonofont[Mapping=tex-text,Scale=0.9]{Inconsolata}


%%%------------------------------------------------------------------------
%%% Local commands
%%%------------------------------------------------------------------------

%% Marginal header
%% Note: as the document goes on you may need to introduce a (gradually increasing)
%% \vspace element to keep the marginal header pleasingly aligned with the first
%% item in the body text. Like this: \marginhead{{\vskip 0.4em}Grants}, or
%% \marginhead{{\vskip 0.8em}Service}. Experiment as needed.
\newcommand{\marginhead}[1]{\marginpar{\textsf{{\normalsize\vspace{-1em}\flushright #1}}}}

\newcommand{\dates}[1]{\hfill \emph{#1}}

%% custom ampersand (font consistent with the one chosen above)
\newcommand{\amper}{{\fontspec[Scale=.95,Colour=AA0000]{Minion Pro Medium}\selectfont\&\,}}

%% No bullets on labels
\renewcommand{\labelitemi}{~}

%% Custom hanging indent for vita items
\def\ind{\hangindent=1 true cm\hangafter=1 \noindent}
%\def\ind{\hangindent=18pt\hangafter=1 \noindent}
\def\labelitemi{~}
\renewcommand{\labelitemii}{~}

%%%------------------------------------------------------------------------
%%% Page layout
%%%------------------------------------------------------------------------
\pagestyle{fancy}
\renewcommand{\headrulewidth}{0pt}
\fancyhead{}
\fancyfoot{}
\rhead{{\scriptsize\thepage}}

% \setlength{\headsep}{12pt}
\textheight=580pt
\raggedbottom
\thispagestyle{fancy}

%% git revision control footer
\rfoot{\texttt{\scriptsize \VCRevision\ on \VCDateTEX}} % git revision info inserted via external script -- see docs for vc package for details. comment out this line if you're not using vc, and also remove the %%% This file has been generated by the vc bundle for TeX.
%%% Do not edit this file!
%%%
%%% Define Git specific macros.
\gdef\GITHash{e0824d783a88eb464395fd836fc8cb9d666bcae5}%
\gdef\GITAbrHash{e0824d7}%
\gdef\GITParentHashes{9523c1e0762ed00ce249602437fcb0106493473d}%
\gdef\GITAbrParentHashes{9523c1e}%
\gdef\GITAuthorName{Heather Miller}%
\gdef\GITAuthorEmail{heather.miller@epfl.ch}%
\gdef\GITAuthorDate{2015-01-27 14:21:31 +0100}%
\gdef\GITCommitterName{Heather Miller}%
\gdef\GITCommitterEmail{heather.miller@epfl.ch}%
\gdef\GITCommitterDate{2015-01-27 14:21:31 +0100}%
%%% Define generic version control macros.
\gdef\VCRevision{\GITAbrHash}%
\gdef\VCAuthor{\GITAuthorName}%
\gdef\VCDateRAW{2015-01-27}%
\gdef\VCDateISO{2015-01-27}%
\gdef\VCDateTEX{2015/01/27}%
\gdef\VCTime{14:21:31 +0100}%
\gdef\VCModifiedText{\textcolor{red}{with local modifications!}}%
%%% Assume clean working copy.
\gdef\VCModified{0}%
\gdef\VCRevisionMod{\VCRevision}%
 line above.


%%%------------------------------------------------------------------------
%%% Address and contact block
%%%------------------------------------------------------------------------
% \begin{absolutelynopagebreak}
\begin{minipage}[t]{2.95in}
 \flushright {\footnotesize \href{http://ic.epfl.ch}{Faculty of Computer, Communication, \\ \vspace{-0.03in} and Information Science} \\ EPFL \\ \vspace{-0.03in} Station 14 \\ \vspace{-0.03in} 1015 Lausanne \\ \vspace{-0.05in} Switzerland}

\end{minipage}
\hfill
%\begin{minipage}[t]{0.0in}
% dummy (needed here)
%\end{minipage}
\hfill
\begin{minipage}[t]{1.7in}
  \flushright \footnotesize Phone: \myphone \\
  Fax: \myfax  \\
  {\scriptsize  \texttt{\href{mailto:\myemail}{\myemail}}} \\
  {\scriptsize  \vspace{-0.03in} \texttt{\href{\myweb}{\myweb}}}
\end{minipage}


\medskip

%% Name
\noindent{\huge {\textsc{heather miller}}}
\reversemarginpar

\medskip

%% Citizenship
\medskip
\marginhead{Citizenship}

\noindent USA

\bigskip


%% Education
\marginhead{Education}

\noindent{\bf \em EPFL}, \emph{Lausanne, Switzerland} \vspace{0.01in} \dates{2009 --}
\newline Ph.D. in Computer Science
\newline Advisor: Martin Odersky \dates{2011 --}
\bigskip

\noindent{\bf \em University of Miami}, \emph{Coral Gables, FL} \vspace{0.01in}  \dates{2006 -- 2009}
\newline\noindent BSEE in Electrical Engineering, Audio Engineering, {\em with honors, May 2009}
\bigskip

\noindent{\bf \em Cooper Union for the Advancement of Science and Art}, \emph{New York, NY} \vspace{0.01in}  \dates{2004 -- 2006}
\bigskip

%% Professional Experience
\medskip
\marginhead{Professional \newline Experience}

\noindent {\bf Research Intern}, {\bf \em Databricks}, \emph{Berkeley, CA, USA} \vspace{0.01in} \dates{8/2014 -- 11/2014}
\newline\noindent Supervisor: Matei Zaharia
\newline\noindent Integrated Scala Pickling, our framework for fast, boilerplate-free, extensible
\newline\noindent serialization focused on distributed programming (OOPSLA'13) into Spark.
\newline\noindent Developed new function-passing programming model and framework, can be
\newline\noindent thought of as a generalization of Spark/MapReduce programming model.
\bigskip

%% Teaching Experience
\medskip
\marginhead{Teaching \newline Experience}

% \begin{minipage}{\linewidth}
\noindent {\bf Lecturer, Co-Designer}, {\em Reactive Programming \& Parallelism} \dates{2015}
\newline\noindent EPFL Undergraduate course on parallel, distributed, and asynchronous
\newline\noindent programming (\textasciitilde90 students)
\bigskip

\noindent {\bf Lecturer, Co-Designer}, {\em Parallel Programming \& Data Analysis} \dates{2015}
\newline\noindent Upcoming Coursera MOOC on parallel, distributed, and asynchronous
\newline\noindent programming.
\bigskip

\noindent {\bf Lead}, {\em Functional Programming Principles in Scala} \dates{2012 -- 2014}
\newline\noindent Popular Coursera MOOC on functional programming in Scala,
\newline\noindent with >200,000 participants to date \& largest completion
\newline\noindent rate for a course its size (\textasciitilde19\%)
\vspace{0.05in}
\begin{easylist}[itemize]
& Lead teaching staff organizing a team of graduate students,
\newline managing content production, designed course exercises
\newline with cloud-hosted grading, production of lecture videos, etc

& Created extensive course analysis with interactive
\newline visualizations; led to a publication at ICSE'14
\end{easylist}
\bigskip

\noindent {\bf (Lead) Teaching Assistant}, {\em Programming Principles} \dates{2011-2014}
\newline\noindent Required EPFL undergraduate course on functional \& logic programming
\newline\noindent (\textasciitilde160 students)
\medskip

\noindent {\bf Instructor}, {\em Scala as a Research Tool} \dates{2013}
\newline\noindent ECOOP Tutorial
\pagebreak
% \bigskip


% \noindent {\bf Teaching Assistant}, {\em Programming Principles} \dates{2011, 2014}
% \newline\noindent Required EPFL Undergraduate course on functional and logic programming
% \newline\noindent (\textasciitilde160 students)
% \bigskip

%% Research Interests
\textheight=580pt
\marginhead{{\vskip 0.3em}Research \newline Interests}
% \medskip

\noindent Concurrent, distributed, data-centric, and data-intensive (big data) programming, from the perspective of programming languages. I work on both theoretical ideas {\em \&} implementations for the Scala programming language which seek to make it easier to build distributed systems.
\bigskip

% Programming language support for concurrent and distributed programming; \\type systems; non-standard uses of types for data-centric programming and big data; language and library design


% \bigskip

%% Publications
\marginhead{Publications}
% \medskip

%% Use revnumerate environment if numbered publications are needed.
%% (Include it above in the preamble).
%% \renewcommand{\labelenumi}{\textsc{a}\theenumi.}
%% \begin{revnumerate}

\noindent\href{https://infoscience.epfl.ch/record/205039}{\bf Distributed Programming via Safe Closure Passing}\dates{PLACES 2015}
\newline\noindent Philipp Haller, Heather Miller
\newline\noindent\emph{Programming Language Approaches to Communication}
\newline\noindent\emph{and Concurrency Centric Systems}
\bigskip

\noindent\href{http://infoscience.epfl.ch/record/191239}{\bf Spores: A Type-Based Foundation for Closures in the Age of}\dates{ECOOP 2014}\vspace{-0.03in}
\newline\noindent\href{http://infoscience.epfl.ch/record/191239}{\bf Concurrency and Distribution}
% \smallskip
\newline\noindent Heather Miller, Philipp Haller, Martin Odersky
\newline\noindent\emph{European Conference on Object Oriented Programming}
\bigskip

\noindent\href{http://infoscience.epfl.ch/record/190022}{\bf Functional Programming For All! Scaling a MOOC for Students}\dates{ICSE 2014}\vspace{-0.03in}
\newline\noindent\href{http://infoscience.epfl.ch/record/190022}{\bf And Professionals Alike}
% \smallskip
\newline\noindent Heather Miller, Philipp Haller, Lukas Rytz, Martin Odersky
\newline\noindent\emph{ACM SIGSOFT International Conference on Software Engineering}
\bigskip

\noindent\href{http://infoscience.epfl.ch/record/188383}{\bf Instant Pickles: Generating Object-Oriented Pickler}\dates{OOPSLA 2013}\vspace{-0.03in}
\newline\noindent\href{http://infoscience.epfl.ch/record/188383}{\bf Combinators for Fast and Extensible Serialization}
% \smallskip
\newline\noindent Heather Miller, Philipp Haller, Eugene Burmako, Martin Odersky
\newline\noindent\emph{ACM SIGPLAN Conference on Object Oriented Programming, Systems,}
\newline\noindent\emph{Languages and Applications}
\bigskip

\noindent\href{http://infoscience.epfl.ch/record/188383}{\bf RAY: Integrating Rx and Async for Direct-Style Reactive Streams}\dates{REM 2013}
% \smallskip
\newline\noindent Philipp Haller, Heather Miller
\newline\noindent\emph{ACM SPLASH Workshop on Reactivity, Events and Modularity}
\bigskip

\noindent\href{http://infoscience.epfl.ch/record/180265}{\bf FlowPools: A Lock-Free Deterministic Concurrent}\dates{LCPC 2012}\vspace{-0.03in}
\newline\noindent\href{http://infoscience.epfl.ch/record/180265}{\bf Dataflow Abstraction}
% \smallskip
\newline\noindent Aleksandar Prokopec, Heather Miller, Tobias Schlatter,
\newline\noindent Philipp Haller, Martin Odersky
\newline\noindent\emph{International Workshop on Languages and Compilers for Parallel Computing}
\vspace{0.03in}
\newline\noindent {\small Invited to Revised Selected Papers on the 25th International Workshop on}
\vspace{-0.03in}
\newline\noindent {\small Languages and Compilers for Parallel Computing, Lecture Notes in Computer}
\vspace{-0.03in}
\newline\noindent {\small Science, Vol. 7760, 2013}
\bigskip

\noindent\href{http://infoscience.epfl.ch/record/170032}{\bf Tools and Frameworks for Big Learning in Scala: Leveraging the}\dates{BigLearn 2011}\vspace{-0.03in}
\newline\noindent\href{http://infoscience.epfl.ch/record/170032}{\bf Language for High Productivity and Performance}
% \smallskip
\newline\noindent Heather Miller, Philipp Haller, Martin Odersky
\newline\noindent\emph{NIPS Workshop on Parallel and Large-Scale Machine Learning}
\bigskip

\noindent\href{http://infoscience.epfl.ch/record/165111}{\bf Parallelizing Machine Learning -- Functionally: A Framework}\dates{Scala 2011}\vspace{-0.03in}
\newline\noindent\href{http://infoscience.epfl.ch/record/165111}{\bf and Abstractions for Parallel Graph Processing}
% \smallskip
\newline\noindent Philipp Haller, Heather Miller
\newline\noindent\emph{Scala Workshop}
% \bigskip

\pagebreak
%% In Progress
\marginhead{{\vskip 0.4em}Submitted/In Preparation}
\medskip

%% Use revnumerate environment if numbered publications are needed.
%% (Include it above in the preamble).
%% \renewcommand{\labelenumi}{\textsc{a}\theenumi.}
%% \begin{revnumerate}

% \noindent\href{http://infoscience.epfl.ch/record/191239}{\bf Function-Passing Style: Typed, Distributed}\dates{}\vspace{-0.03in}
% \newline\noindent\href{http://infoscience.epfl.ch/record/191239}{\bf Functional Programming}
\noindent{\bf Function-Passing Style: Typed, Distributed Functional Programming}\dates{}
% \newline\noindent{\bf Functional Programming}
% \smallskip
\newline\noindent Heather Miller, Philipp Haller
% \bigskip
\medskip

\noindent{\bf Self-Assembly: Lightweight Language Extension and Datatype Generic Programming, All-in-One!}\dates{}
\newline\noindent Heather Miller, Philipp Haller, Bruno C. d. S. Oliveira
% \bigskip
\medskip

\noindent{\bf Improving Human-Compiler Interaction Through Customizable Type Feedback}\dates{}
\newline\noindent Hubert Plociniczak, Heather Miller, Martin Odersky
\bigskip

%% Selected Tech Reports
\marginhead{{\vskip 0.3em}Selected \newline Tech Reports}
% \medskip

%% Use revnumerate environment if numbered publications are needed.
%% (Include it above in the preamble).
%% \renewcommand{\labelenumi}{\textsc{a}\theenumi.}
%% \begin{revnumerate}

\noindent\href{http://infoscience.epfl.ch/record/191240}{\bf Spores, Formally}\dates{}
% \smallskip
\newline\noindent Heather Miller, Philipp Haller
\newline\noindent\emph{December 2013}
% \bigskip
\medskip

\noindent\href{http://infoscience.epfl.ch/record/181098}{\bf  FlowPools: A Lock-Free Deterministic Concurrent Dataflow Abstraction -- Proofs}\dates{}
% \smallskip
\newline\noindent Aleksandar Prokopec, Heather Miller, Philipp Haller
\newline\noindent\emph{June 2012}
% \bigskip
\medskip

%% Open Source
\medskip
\marginhead{{\vskip 0.1em}Open Source}

\vspace{0.01in}
\noindent {\bf Scala Programming Language}, {\em member of the Scala team} \dates{2011 --}

\vspace{0.05in}
\begin{easylist}[itemize]
& \href{http://docs.scala-lang.org/sips/pending/spores.html}{{\bf Scala Spores} (Scala Improvement Proposal SIP-21)}, {\bf \em project lead}
\newline novel type-based abstraction for using closures safely
\newline in concurrent and distributed environments

& \href{http://lampwww.epfl.ch/~hmiller/pickling/}{{\bf Scala Pickling}}, {\bf \em project lead}
\newline novel framework for fast, boilerplate-free, extensible serialization.
\newline Adopted by sbt, the most widely-used build tool for Scala. Popular
\newline open-source project on GitHub with >480 stars \& dozens of contributors

& \href{http://docs.scala-lang.org/sips/completed/futures-promises.html}{{\bf Scala Futures \& Promises} (Scala Improvement Proposal SIP-14)}, {\bf \em team member}
\newline unified non-blocking concurrency substrate for
\newline Scala, Akka, Play, and others

& \href{http://docs.scala-lang.org/}{{\bf Scala Documentation}}, {\bf \em creator, writer, lead maintainer}
\newline a central website for community-driven documentation for
\newline the Scala programming language and core libraries

& \href{https://wiki.scala-lang.org/display/SW/Scaladoc}{{\bf Scaladoc}}, {\bf \em co-maintainer}
\newline documentation tool for Scala's official API documentation

\end{easylist}

\bigskip

%% Honors
\medskip
\marginhead{Honors}

\noindent US National Science Foundation Graduate Research Fellowship \dates{2011 -- 2014}
% \newline\noindent PLMW Travel Grant \dates{2014-2015}
% \newline\noindent ICSE Travel Grant \dates{2014}
\newline\noindent EPFL Outstanding Teaching Award \dates{2012}
\newline\noindent EPFL Computer Science Fellowship \dates{2009 -- 2010}
\newline\noindent Most Outstanding Audio Engineering Student, University of Miami \dates{2009}
\newline\noindent Most Outstanding Eta Kappa Nu Student, University of Miami \dates{2009}
\newline\noindent Information Technology Scholarship, University of Miami \dates{2006 -- 2009}
\newline\noindent John Farina Family Scholarship, University of Miami \dates{2006 -- 2009}
\newline\noindent Eta Kappa Nu \dates{2008}
\newline\noindent Tau Beta Pi \dates{2008}
\newline\noindent SMART US Department of Defense Scholarship Alternate \dates{2007}
\newline\noindent Cooper Union Full Tuition Scholarship \dates{2004 -- 2006}

\bigskip

\pagebreak
%% Talks
\medskip
\marginhead{Selected Talks}

\vspace{-0.02in}
\noindent\href{https://speakerdeck.com/heathermiller/function-passing-style-typed-distributed-functional-programming}{\bf Function Passing Style: Typed, Distributed} \dates{Strange Loop 2014}\vspace{-0.03in}
\linebreak\noindent\href{https://speakerdeck.com/heathermiller/function-passing-style-typed-distributed-functional-programming}{\bf Functional Programming}\dates{}
\linebreak\noindent St. Louis, MO, USA. September 19, 2014
\bigskip


\noindent\href{https://speakerdeck.com/heathermiller/spores-a-type-based-foundation-for-closures-in-the-age-of-concurrency-and-distribution}{\bf Spores: A Type-Based Foundation for Closures in the Age of} \dates{ECOOP 2014}\vspace{-0.03in}
\linebreak\noindent\href{https://speakerdeck.com/heathermiller/spores-a-type-based-foundation-for-closures-in-the-age-of-concurrency-and-distribution}{\bf Concurrency and Distribution}\dates{}
\linebreak\noindent Uppsala, Sweden. August 1, 2014
\bigskip

\noindent{\bf Functional Programming For All! Scaling a MOOC for} \dates{ICSE 2014}\vspace{-0.03in}
\linebreak\noindent{\bf Students and Professionals Alike}\dates{}
\linebreak\noindent Hyderabad, India. June 4, 2014
\bigskip

\noindent{\bf Academese to English: Scala's Type System, Dependent Types} \dates{NEScala 2014}\vspace{-0.03in}
\linebreak\noindent{\bf and What It Means To You}\dates{}
\linebreak\noindent New York, NY, USA. March 1, 2014
\bigskip

\noindent\href{https://speakerdeck.com/heathermiller/instant-pickles-generating-object-oriented-pickler-combinators-for-fast-and-extensible-serialization}{\bf Instant Pickles: Generating Object-Oriented Pickler} \dates{OOPSLA 2013}\vspace{-0.03in}
\linebreak\noindent\href{https://speakerdeck.com/heathermiller/instant-pickles-generating-object-oriented-pickler-combinators-for-fast-and-extensible-serialization}{\bf Combinators for Fast and Extensible Serialization}\dates{}
\linebreak\noindent Indianapolis, IN, USA. October 30, 2013
\bigskip

\noindent\href{http://heather.miller.am/files/IU-PL-Abstractions-for-Dist-Programming.pdf}{\bf PL Abstractions for Distributed Programming:} \dates{Indiana University {\bf \em (invited)}}\vspace{-0.03in}
\linebreak\noindent\href{http://heather.miller.am/files/IU-PL-Abstractions-for-Dist-Programming.pdf}{\bf Pickle Your Spores!}\dates{}
\linebreak\noindent Bloomington, IN, USA. October 25, 2013
\bigskip

\noindent\href{https://speakerdeck.com/heathermiller/spores-distributable-functions-in-scala}{\bf Spores: Distributable Functions in Scala} \dates{Strange Loop 2013}
\linebreak\noindent St. Louis, MO, USA. September 19, 2013
\bigskip

\noindent\href{http://heather.miller.am/files/LaME2013-Dataflow.pdf}{\bf Open Issues in Dataflow Programming} \dates{LaME 2013 {\bf \em (invited)}}
\linebreak\noindent Montpellier, France. July 1, 2013
\bigskip

\noindent{\bf Scala as a Research Tool} \dates{ECOOP 2013 Tutorial}
\linebreak\noindent Montpellier, France. July 1, 2013
\bigskip

\noindent\href{https://speakerdeck.com/heathermiller/on-pickles-and-spores-improving-support-for-distributed-programming-in-scala}{\bf On Pickles \& Spores: Improving Scala's Support} \dates{ScalaDays 2013}\vspace{-0.03in}
\linebreak\noindent\href{https://speakerdeck.com/heathermiller/on-pickles-and-spores-improving-support-for-distributed-programming-in-scala}{\bf for Distributed Programming}\dates{}
\linebreak\noindent New York, NY, USA. June 12, 2013
\bigskip

\noindent\href{http://lampwww.epfl.ch/~hmiller/files/Futures-Try-PhillyETE.pdf}{\bf Futures \& Promises in Scala 2.10} \dates{PhillyETE 2013 {\bf \em (invited)}}
\linebreak\noindent Philadelphia, PA, USA. April 2, 2013
\bigskip

\bigskip
\noindent {\em I am also a frequent speaker in industry, at industrial conferences, developer ``meet-ups'', and everything in between. Some such events include:}
\vspace{0.3em}
\newline\noindent
{\bf\href{http://fby.by/}{f(by)}} (11/2014, Minsk, Belarus),
{\bf\href{https://www.youtube.com/watch?v=4obTnLVXQWY}{SF Scala}} (11/2014, SF, USA),
{\bf\href{http://www.scalapeno.org.il/#!heather-miller/cj0q}{Scalape\~{n}o}} (9/2014, Tel Aviv, Israel),
{\bf \href{https://www.eventbrite.com/e/soundcloud-techtalks-unconventional-thinking-in-design-and-programming-tickets-12166429117}{SoundCloud TechTalks}} (7/2014, Berlin, Germany),
{\bf Scala Days} (6/2014, Berlin, Germany),
{\bf\href{http://www.nescala.org/2014}{NEScala}} (3/2014, NYC, USA), amongst others.

\bigskip


% %% Selected Broader Service
% \medskip
% \marginhead{Selected \newline Broader \newline Service}

% \noindent \href{http://ic.epfl.ch/conseil-de-faculte}{\bf EPFL Computer Science Faculty Council}, {\bf \em PhD Student Representative} \dates{2012 --}
% \newline\noindent Members include the dean of the faculty as well as representatives
% \newline\noindent from every branch of the faculty, administrative, PhD, faculty, etc.
% \newline\noindent Quarterly meetings to steer the faculty and introduce new initiatives.
% \bigskip

% \noindent \href{http://ic-gsa.epfl.ch/}{\bf EPFL CS Graduate Student Association}, {\bf \em President} \dates{2009 -- 2011}
% \newline\noindent Volunteer student organization with a mission to foster a sense of
% \newline\noindent community and collaboration between different research groups in
% \newline\noindent the faculty. Initiatives led/introduced:
% \vspace{0.05in}
% \begin{easylist}[itemize]
% & {\bf Research Day}: college-wide showcase of labs' research activities
% & {\bf PhD Student Open House}: main recruiting event for CS doctoral program
% & {\bf Social Events}: aper\'{o}s, ski trips, outings
% \end{easylist}
% \bigskip

% \noindent {\bf EPFL CS Graduate Student Mentor} \dates{2010 -- 2012}
% \newline\noindent One-on-one mentoring of incoming doctoral students, aided students in
% \newline\noindent integrating into EPFL's research environment and Switzerland as a whole.
% \vspace{0.05in}
% \bigskip


%% External Activities
\medskip
\marginhead{External \newline Activities}

\noindent {\bf \href{https://www.hackerschool.com/}{Hacker School}}, resident\dates{2015}
\newline\noindent {\bf \href{http://scalawags.tv/}{Scalawags Monthly Podcast}}, co-host\dates{2014 --}

\bigskip

%% External Service
\medskip
\marginhead{External \newline Service}

% \noindent {\bf Committees:}
\noindent\newline\noindent {\bf Curry On 2015}, organizer (co-chair)\dates{7/2015}
\newline\noindent {\bf ECOOP 2015}, organizing committee member (sponsorship)\dates{7/2015}
\newline\noindent {\bf PLE 2015}, program committee member\dates{7/2015}
\newline\noindent {\bf DSLDI 2015}, program committee member\dates{7/2015}
\newline\noindent {\bf Scala Symposium 2015}, organizer (co-chair)\dates{6/2015}
\newline\noindent {\bf POPL 2015}, artifact evaluation committee member\dates{1/2015}
\newline\noindent {\bf Scala Workshop 2014}, organizer (co-chair)\dates{7/2014}
\newline\noindent {\bf Scala Workshop 2013}, organizer (co-chair)\dates{7/2013}

\medskip
\noindent {\bf External Reviewer} for: ECOOP 2013, Scala 2013
\newline\noindent {\bf Editor of proceedings} for: Scala 2015, Scala 2014, Scala 2013

\bigskip

% %% Service
% \marginhead{Academic \newline Service}

% \vspace{-0.02in}
% \noindent{\bf Committees}: Curry On Prague (co-chair), Scala 2015 (co-chair), ECOOP 2015 organizing committee (sponsorship chair), POPL 2015 AEC, Scala 2014 (co-chair), Scala 2013 (co-chair)
% \newline\noindent{\bf Reviewer} for: ECOOP 2013, Scala 2013

% \bigskip

%% Students Supervised
\medskip
\marginhead{Students \newline Supervised\footnotemark[1]}
\footnotetext[1]{At EPFL, research groups offer substantial projects for B.Sc./M.Sc. students to complete for credit. EPFL PhD students design and supervise these projects, as well as M.Sc. thesis projects.}

\noindent {\bf Louis Bliss}, {\em Incremental Picklers for Scala Pickling} \dates{9/2013 -- 1/2014}
\newline\noindent M.Sc. level, co-supervision with Philipp Haller
\medskip

\noindent {\bf Thadd\'{e}e Yann Tyl}, {\em Learning Scala Style} \dates{2/2013 -- 6/2013}
\newline\noindent M.Sc. thesis
\medskip

\noindent {\bf Tobias Schlatter}, {\em FlowSeqs: Barrier-Free ParSeqs} \dates{9/2012 -- 1/2013}
\newline\noindent M.Sc. level, co-supervision w/ Philipp Haller \& Aleksandar Prokopec
\medskip

\noindent {\bf Tobias Schlatter}, {\em Multi-Lane FlowPools} \dates{2/2012 -- 6/2012}
\newline\noindent M.Sc. level, co-supervision w/ Philipp Haller \& Aleksandar Prokopec
\medskip

\noindent {\bf Pierre Grydbeck}, {\em Parallel Machine Learning: An Expectation} \dates{2/2012 -- 6/2012}
\newline\noindent {\em Maximization Algorithm for Gaussian Mixture Models}
\newline\noindent M.Sc. level, co-supervision with Philipp Haller
\medskip

\noindent {\bf Bruno Studer}, {\em Parallel Machine Learning: Collaborative Filtering} \dates{2/2012 -- 6/2012}
\newline\noindent {\em via Alternating Least Squares}
\newline\noindent B.Sc. level, co-supervision with Philipp Haller
\medskip

\noindent {\bf Stanislav Peshterliev}, {\em Parallel Natural Language Processing} \dates{9/2011 -- 1/2012}
\newline\noindent {\em Algorithms in Scala}
\newline\noindent M.Sc. level, co-supervision with Philipp Haller
\medskip

\noindent {\bf Olivier Blanvillain \& Louis Bliss}, {\em Parallelization of a Collaborative} \dates{9/2011 -- 1/2012}
\newline\noindent {\em Filtering Algorithm with Menthor}
\newline\noindent B.Sc. level, co-supervision with Philipp Haller
\medskip

\noindent {\bf Florian Gysin}, {\em Improving Parallel Graph Processing Through} \dates{9/2011 -- 1/2012}
\newline\noindent {\em the Introduction of Parallel Collections}
\newline\noindent M.Sc. level, co-supervision with Philipp Haller
\medskip

\noindent {\bf Georges Discry}, {\em Extending the Menthor Framework for Parallel} \dates{2/2011 -- 6/2011}
\newline\noindent {\em Graph Processing to Distributed Computing}
\newline\noindent M.Sc. level, co-supervision with Philipp Haller
\medskip

\pagebreak
%% References
\medskip
\marginhead{References}

\noindent {\bf Martin Odersky}
\newline\noindent {Faculty of Computer, Communication, and Information Science}
\newline\noindent {\em \'{E}cole Polytechnique F\'{e}d\'{e}rale de Lausanne}
\newline\noindent \Telefon~+41 21 693 68 63
\newline\noindent \Letter~\href{mailto:martin.odersky@epfl.ch}{martin.odersky@epfl.ch}
\medskip

\noindent {\bf Philipp Haller}
\newline\noindent {School of Computer Science and Communication}
\newline\noindent {\em KTH Royal Institute of Technology}
\newline\noindent \Telefon~+41 76 205 39 32
\newline\noindent \Letter~\href{mailto:phaller@kth.se}{phaller@kth.se}
\medskip

\noindent {\bf Matei Zaharia}
\newline\noindent {Department of Electrical Engineering and Computer Science}
\newline\noindent {\em Massachusetts Institute of Technology}
\newline\noindent \Telefon~+1-510-610-0001
\newline\noindent \Letter~\href{mailto:matei@mit.edu}{matei@mit.edu}
\medskip

\noindent {\bf Marius Eriksen}
\newline\noindent {\em Twitter}
% \newline\noindent \Telefon~+41 76 205 39 32
\newline\noindent \Letter~\href{mailto:marius@twitter.com}{marius@twitter.com}
\medskip



\end{document}

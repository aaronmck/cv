%%% A template to produce a nice-looking Curriculum Vitae.
%%% Original by Kieran Healy <kjhealy@gmail.com>, tweaked by Heather Miller <heather.c.miller@gmail.com>
%%% Most recent version is at http://github.com/heathermiller/cv
%%%
%%% ------------------------------------------------------------------------
%%% Requirements (should be included in a modern tex distribution):
%%% ------------------------------------------------------------------------
%%% xelatex
%%% fontspec.sty
%%% hyperrref.sty
%%% xunicode.sty
%%% color.sty
%%% url.sty
%%% fancyhdr.sty
%%%
%%% ------------------------------------------------------------------------
%%% Optional
%%% ------------------------------------------------------------------------
%%% git
%%% vc.sty
%%% revnum.sty
%%% Fonts
%%%
%%% ------------------------------------------------------------------------
%%% Note
%%%------------------------------------------------------------------------
%%% Because this is a hand-tweaked file, be on the look out for \medksip,
%%% \bigskip and \newpage commands here and there, which are used to balance
%%% the layout or avoid widows & orphans, etc. You should of course add or
%%% remove these as needed.
%%%------------------------------------------------------------------------

\documentclass[9pt]{article}

%%%------------------------------------------------------------------------
%%% Metadata
%%%------------------------------------------------------------------------

%% Change as needed. Or just add me as a coauthor. Only some of these are
%% used below in the hyperref declaration and address banner section.
\def\myauthor{Heather Miller}
\def\mytitle{Vita}
\def\mycopyright{\myauthor}
\def\mykeywords{}
\def\mybibliostyle{plain}
\def\mybibliocommand{}
\def\mysubtitle{}
\def\myaffiliation{EPFL}
\def\myaddress{Faculty of Computer, Communication, and Information Sciences}
\def\myemail{heather.miller@epfl.ch}
\def\myweb{http://heather.miller.am}
\def\myfax{+41 21 693 66 60}
\def\myphone{+41 78 625 20 23}
\def\myversion{}
\def\myrevision{}


\def\myaffiliation{EPFL}
\def\myauthor{Heather Miller}
\date{} % not used (revision control instead)
\def\mykeywords{Heather, Miller, Heather Miller, Vita, CV, Resume, Scala, Programming Languages}

%%%------------------------------------------------------------------------
%%% Git version tracking
%%%------------------------------------------------------------------------

%% If you don't use git or the vc package (from CTAN), comment this out.
%% If you comment it out, be sure to remove the \rfoot comment below, too.
\immediate\write18{sh ./vc}
%%% This file has been generated by the vc bundle for TeX.
%%% Do not edit this file!
%%%
%%% Define Git specific macros.
\gdef\GITHash{e0824d783a88eb464395fd836fc8cb9d666bcae5}%
\gdef\GITAbrHash{e0824d7}%
\gdef\GITParentHashes{9523c1e0762ed00ce249602437fcb0106493473d}%
\gdef\GITAbrParentHashes{9523c1e}%
\gdef\GITAuthorName{Heather Miller}%
\gdef\GITAuthorEmail{heather.miller@epfl.ch}%
\gdef\GITAuthorDate{2015-01-27 14:21:31 +0100}%
\gdef\GITCommitterName{Heather Miller}%
\gdef\GITCommitterEmail{heather.miller@epfl.ch}%
\gdef\GITCommitterDate{2015-01-27 14:21:31 +0100}%
%%% Define generic version control macros.
\gdef\VCRevision{\GITAbrHash}%
\gdef\VCAuthor{\GITAuthorName}%
\gdef\VCDateRAW{2015-01-27}%
\gdef\VCDateISO{2015-01-27}%
\gdef\VCDateTEX{2015/01/27}%
\gdef\VCTime{14:21:31 +0100}%
\gdef\VCModifiedText{\textcolor{red}{with local modifications!}}%
%%% Assume clean working copy.
\gdef\VCModified{0}%
\gdef\VCRevisionMod{\VCRevision}%


%%%------------------------------------------------------------------------
%%% Required style files
%%%------------------------------------------------------------------------

\usepackage{url,fancyhdr}
%%\usepackage{revnum} % for reverse-numbered publications (revnumerate environment) if needed.

%% needed for xelatex to work
\usepackage{fontspec}
\usepackage{xunicode}

%% color for the links
% \usepackage[usenames,dvipsnames]{color}
\usepackage[usenames,dvipsnames]{xcolor}
\definecolor{DarkBlue}{HTML}{265B8C}

%% hyperlinks
\usepackage[xetex,
	colorlinks=true,
	urlcolor=DarkBlue,
	plainpages=false,
  	pdfpagelabels,
  	bookmarksnumbered,
  	pdftitle={\mytitle},
  	pagebackref,
  	pdfauthor={\myauthor},
  	pdfkeywords={\mykeywords}
  	]{hyperref}

%%%------------------------------------------------------------------------
%%% Document
%%%------------------------------------------------------------------------
\begin{document}

%% Choose fonts for use with xelatex
%% Minion and Myriad are widely available, from Adobe.
%% Pragmata is available to buy at http://www.fsd.it/fonts/pragma.htm
%% and is worth every penny. Any good monospace font will work fine, though.
%% Consolas or inconsolata are good alternatives.
\setromanfont[Mapping={tex-text},Numbers={OldStyle},Ligatures={Common}]{Minion Pro}
\setsansfont[Mapping=tex-text,Colour=AA0000]{Myriad Pro}
\setmonofont[Mapping=tex-text,Scale=0.9]{Inconsolata}


%%%------------------------------------------------------------------------
%%% Local commands
%%%------------------------------------------------------------------------

%% Marginal header
%% Note: as the document goes on you may need to introduce a (gradually increasing)
%% \vspace element to keep the marginal header pleasingly aligned with the first
%% item in the body text. Like this: \marginhead{{\vskip 0.4em}Grants}, or
%% \marginhead{{\vskip 0.8em}Service}. Experiment as needed.
\newcommand{\marginhead}[1]{\marginpar{\textsf{{\footnotesize\vspace{-1em}\flushright #1}}}}

\newcommand{\dates}[1]{\hfill \emph{#1}}

%% custom ampersand (font consistent with the one chosen above)
\newcommand{\amper}{{\fontspec[Scale=.95,Colour=AA0000]{Minion Pro Medium}\selectfont\&\,}}

%% No bullets on labels
\renewcommand{\labelitemi}{~}

%% Custom hanging indent for vita items
\def\ind{\hangindent=1 true cm\hangafter=1 \noindent}
%\def\ind{\hangindent=18pt\hangafter=1 \noindent}
\def\labelitemi{~}
\renewcommand{\labelitemii}{~}

%%%------------------------------------------------------------------------
%%% Page layout
%%%------------------------------------------------------------------------
\pagestyle{fancy}
\renewcommand{\headrulewidth}{0pt}
\fancyhead{}
\fancyfoot{}
\rhead{{\scriptsize\thepage}}

%% git revision control footer
\rfoot{\texttt{\scriptsize \VCRevision\ on \VCDateTEX}} % git revision info inserted via external script -- see docs for vc package for details. comment out this line if you're not using vc, and also remove the %%% This file has been generated by the vc bundle for TeX.
%%% Do not edit this file!
%%%
%%% Define Git specific macros.
\gdef\GITHash{e0824d783a88eb464395fd836fc8cb9d666bcae5}%
\gdef\GITAbrHash{e0824d7}%
\gdef\GITParentHashes{9523c1e0762ed00ce249602437fcb0106493473d}%
\gdef\GITAbrParentHashes{9523c1e}%
\gdef\GITAuthorName{Heather Miller}%
\gdef\GITAuthorEmail{heather.miller@epfl.ch}%
\gdef\GITAuthorDate{2015-01-27 14:21:31 +0100}%
\gdef\GITCommitterName{Heather Miller}%
\gdef\GITCommitterEmail{heather.miller@epfl.ch}%
\gdef\GITCommitterDate{2015-01-27 14:21:31 +0100}%
%%% Define generic version control macros.
\gdef\VCRevision{\GITAbrHash}%
\gdef\VCAuthor{\GITAuthorName}%
\gdef\VCDateRAW{2015-01-27}%
\gdef\VCDateISO{2015-01-27}%
\gdef\VCDateTEX{2015/01/27}%
\gdef\VCTime{14:21:31 +0100}%
\gdef\VCModifiedText{\textcolor{red}{with local modifications!}}%
%%% Assume clean working copy.
\gdef\VCModified{0}%
\gdef\VCRevisionMod{\VCRevision}%
 line above.

%%%------------------------------------------------------------------------
%%% Address and contact block
%%%------------------------------------------------------------------------
\begin{minipage}[t]{2.95in}
 \flushright {\footnotesize \href{http://ic.epfl.ch}{Faculty of Computer, Communication, \\ \vspace{-0.03in} and Information Science} \\ EPFL \\ \vspace{-0.03in} Station 14 \\ \vspace{-0.03in} 1015 Lausanne \\ \vspace{-0.05in} Switzerland}

\end{minipage}
\hfill
%\begin{minipage}[t]{0.0in}
% dummy (needed here)
%\end{minipage}
\hfill
\begin{minipage}[t]{1.7in}
  \flushright \footnotesize Phone: \myphone \\
  Fax: \myfax  \\
  {\scriptsize  \texttt{\href{mailto:\myemail}{\myemail}}} \\
  {\scriptsize  \vspace{-0.03in} \texttt{\href{\myweb}{\myweb}}}
\end{minipage}


\medskip

%% Name
\noindent{\Large {\textsc{heather miller}}}
\reversemarginpar

\medskip

%% Research Interests
\medskip
\marginhead{Citizenship}

\noindent USA

\bigskip

%% Research Interests
\marginhead{Research \newline Interests}

\noindent Programming language design and implementation for distributed programming. I'm interested in using type systems to facilitate the design of new, functional distributed systems.

\bigskip

%% Education
\marginhead{Education}

\noindent{\bf \em EPFL}, \emph{Lausanne, Switzerland} \vspace{0.01in}  \dates{2009 --}

\ind Ph.D. in Computer Science

\ind Advisor: \ind Martin Odersky \dates{2011 --}

\bigskip

\noindent{\bf \em University of Miami}, \emph{Coral Gables, FL} \vspace{0.01in}  \dates{2006 -- 2009}

\ind BSEE in Electrical Engineering, Audio Engineering, {\em with honors, May 2009}

\bigskip

\noindent{\bf \em Cooper Union for the Advancement of Science and Art}, \emph{New York, NY} \vspace{0.01in}  \dates{2004 -- 2006}

\bigskip

%% Service
\marginhead{Service}

\noindent{\bf \em Committees}: Scala 2014 (co-chair), Scala 2013 (co-chair)
\medskip

\noindent{\bf \em Reviewer} for: ECOOP 2013

\bigskip

%% Publications
\marginhead{{\vskip 0.3em}Publications}
\medskip

%% Use revnumerate environment if numbered publications are needed.
%% (Include it above in the preamble).
%% \renewcommand{\labelenumi}{\textsc{a}\theenumi.}
%% \begin{revnumerate}

\noindent\href{http://infoscience.epfl.ch/record/190022}{\bf Functional Programming For All! Scaling a MOOC for Students}\dates{ICSE 2014}\vspace{-0.03in}
\newline\noindent\href{http://infoscience.epfl.ch/record/190022}{\bf And Professionals Alike}
% \smallskip
\newline\noindent Heather Miller, Philipp Haller, Lukas Rytz, Martin Odersky
\newline\noindent\emph{ACM SIGSOFT International Conference on Software Engineering}
\bigskip

\noindent\href{http://infoscience.epfl.ch/record/188383}{\bf RAY: Integrating Rx and Async for Direct-Style Reactive Streams}\dates{REM 2013}
% \smallskip
\newline\noindent Philipp Haller, Heather Miller
\newline\noindent\emph{Workshop on Reactivity, Events and Modularity}
\bigskip

\noindent\href{http://infoscience.epfl.ch/record/188383}{\bf Instant Pickles: Generating Object-Oriented Pickler}\dates{OOPSLA 2013}\vspace{-0.03in}
\newline\noindent\href{http://infoscience.epfl.ch/record/188383}{\bf Combinators for Fast and Extensible Serialization}
% \smallskip
\newline\noindent Heather Miller, Philipp Haller, Eugene Burmako, Martin Odersky
\newline\noindent\emph{ACM SIGPLAN Conference on Object Oriented Programming, Systems,}
\newline\noindent\emph{Languages and Applications}
\bigskip


\noindent\href{http://infoscience.epfl.ch/record/180265}{\bf FlowPools: A Lock-Free Deterministic Concurrent}\dates{LCPC 2012}\vspace{-0.03in}
\newline\noindent\href{http://infoscience.epfl.ch/record/180265}{\bf Dataflow Abstraction}
% \smallskip
\newline\noindent Aleksandar Prokopec, Heather Miller, Tobias Schlatter,
\newline\noindent Philipp Haller, Martin Odersky
\newline\noindent\emph{International Workshop on Languages and Compilers for Parallel Computing}
\bigskip

\noindent\href{http://infoscience.epfl.ch/record/170032}{\bf Tools and Frameworks for Big Learning in Scala: Leveraging the}\dates{BigLearn 2011}\vspace{-0.03in}
\newline\noindent\href{http://infoscience.epfl.ch/record/170032}{\bf Language for High Productivity and Performance}
% \smallskip
\newline\noindent Heather Miller, Philipp Haller, Martin Odersky
\newline\noindent\emph{NIPS Workshop on Parallel and Large-Scale Machine Learning}
\bigskip

\noindent\href{http://infoscience.epfl.ch/record/165111}{\bf Parallelizing Machine Learning -- Functionally: A Framework}\dates{Scala 2011}\vspace{-0.03in}
\newline\noindent\href{http://infoscience.epfl.ch/record/165111}{\bf and Abstractions for Parallel Graph Processing}
% \smallskip
\newline\noindent Philipp Haller, Heather Miller
\newline\noindent\emph{Scala Workshop}
\bigskip

% \bigskip

%% Awards
\medskip
\marginhead{Awards}

\noindent US National Science Foundation Graduate Research Fellowship \dates{2011 -- 2014}
\newline\noindent EPFL Outstanding Teaching Award \dates{2012}
\newline\noindent EPFL Computer Science Fellowship \dates{2009 -- 2010}
\newline\noindent Most Outstanding Audio Engineering Student, University of Miami \dates{2009}
\newline\noindent Most Outstanding Eta Kappa Nu Student, University of Miami \dates{2009}
\newline\noindent Information Technology Scholarship, University of Miami \dates{2006 -- 2009}
\newline\noindent John Farina Family Scholarship, University of Miami \dates{2006 -- 2009}
\newline\noindent Eta Kappa Nu \dates{2008}
\newline\noindent Tau Beta Pi \dates{2008}
\newline\noindent SMART US Department of Defense Scholarship Alternate \dates{2007}
\newline\noindent Cooper Union Full Tuition Scholarship \dates{2004 -- 2006}

\bigskip

%% Teaching Experience
\medskip
\marginhead{Teaching \newline Experience}

\noindent {\bf Lead Teaching Assistant}, {\em Functional Programming Principles in Scala} \dates{2012 -- 2014}
\newline\noindent Popular Coursera MOOC on functional programming in Scala,
\newline\noindent with >100,000 participants to date
% \newline\noindent {\em Led the development of all aspects of the MOOC}
\bigskip

\noindent {\bf Instructor}, {\em Scala as a Research Tool} \dates{2013}
\newline\noindent ECOOP Tutorial
\bigskip

\noindent {\bf Lead Teaching Assistant}, {\em Programming Principles} \dates{2012}
\newline\noindent EPFL Undergraduate course on functional and logic programming
\bigskip

\noindent {\bf Teaching Assistant}, {\em Programming Principles} \dates{2011}
\newline\noindent EPFL Undergraduate course on functional and logic programming

\bigskip

%% Open Source
\medskip
\marginhead{Open Source}

\noindent {\bf Scala Programming Language}, {\em member of the Scala team} \dates{2011 --}
\newline\ind Scala Documentation Czar
\newline\ind Futures and Promises Library
\newline\ind Scala Pickling
% \begin{itemize}
% \item Scala Documentation Czar
% \end{itemize}

\bigskip

%% Talks
\medskip
\marginhead{Selected Talks}

\noindent{\bf Academese to English: Scala's Type System, Dependent Types} \dates{NEScala 2014}\vspace{-0.03in}
\linebreak\noindent{\bf and What It Means To You}\dates{}
\linebreak\noindent New York, NY, USA. March 1, 2014
\bigskip

\noindent\href{https://speakerdeck.com/heathermiller/instant-pickles-generating-object-oriented-pickler-combinators-for-fast-and-extensible-serialization}{\bf Instant Pickles: Generating Object-Oriented Pickler} \dates{OOPSLA 2013}\vspace{-0.03in}
\linebreak\noindent\href{https://speakerdeck.com/heathermiller/instant-pickles-generating-object-oriented-pickler-combinators-for-fast-and-extensible-serialization}{\bf Combinators for Fast and Extensible Serialization}\dates{}
\linebreak\noindent Indianapolis, IN, USA. October 30, 2013
\bigskip

\noindent\href{http://heather.miller.am/files/IU-PL-Abstractions-for-Dist-Programming.pdf}{\bf PL Abstractions for Distributed Programming:} \dates{Indiana University {\bf \em (invited)}}\vspace{-0.03in}
\linebreak\noindent\href{http://heather.miller.am/files/IU-PL-Abstractions-for-Dist-Programming.pdf}{\bf Pickle Your Spores!}\dates{}
\linebreak\noindent Blooimington, IN, USA. October 25, 2013
\bigskip


\noindent\href{https://speakerdeck.com/heathermiller/spores-distributable-functions-in-scala}{\bf Spores: Distributable Functions in Scala} \dates{Strange Loop 2013}
\linebreak\noindent St. Louis, MO, USA. September 19, 2013
\bigskip

\noindent\href{http://heather.miller.am/files/LaME2013-Dataflow.pdf}{\bf Open Issues in Dataflow Programming} \dates{LaME 2013 {\bf \em (invited)}}
\linebreak\noindent Montpellier, France. July 1, 2013
\bigskip

\noindent{\bf Scala as a Research Tool} \dates{ECOOP 2013 Tutorial}
\linebreak\noindent Montpellier, France. July 1, 2013
\bigskip

\noindent\href{https://speakerdeck.com/heathermiller/on-pickles-and-spores-improving-support-for-distributed-programming-in-scala}{\bf On Pickles \& Spores: Improving Scala's Support} \dates{ScalaDays 2013}\vspace{-0.03in}
\linebreak\noindent\href{https://speakerdeck.com/heathermiller/on-pickles-and-spores-improving-support-for-distributed-programming-in-scala}{\bf for Distributed Programming}\dates{}
\linebreak\noindent New York, NY, USA. June 12, 2013
\bigskip

\noindent\href{http://lampwww.epfl.ch/~hmiller/files/Futures-Try-PhillyETE.pdf}{\bf Futures \& Promises in Scala 2.10} \dates{PhillyETE 2013 {\bf \em (invited)}}
\linebreak\noindent Philadelphia, PA, USA. April 2, 2013
\bigskip

\end{document}
